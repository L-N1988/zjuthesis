\cleardoublepage

\section{绪论}

\subsection{背景}
壬戌之秋,七月既望,苏子与客泛舟游于赤壁之下。清风徐来,水波不兴,举酒属客,诵明月之诗,歌窈窕之章。少焉,月出于东山之上,徘徊于斗牛之间,白露横江,水光接天;纵一苇之所如,凌万顷之茫然。浩浩乎如冯虚御风,而不知其所止;飘飘乎如遗世独立,羽化而登仙。

于是饮酒乐甚,扣舷而歌之。歌曰:“桂棹兮兰桨,击空明兮溯流光。渺渺兮予怀,望美人兮天一方。”客有吹洞箫者,倚歌而和之,其声呜呜然,如怨如慕,如泣如诉,馀音袅袅,不绝如缕。舞幽壑之潜蛟,泣孤舟之嫠妇。

苏子愀然,正襟危坐,而问客曰:“何为其然也?”

客曰:“‘月明星稀,乌鹊南飞’,此非曹孟德之诗乎?西望夏口,东望武昌,山川相缪,郁乎苍苍,此非孟德之困于周郎者乎?方其破荆州,下江陵,顺流而东也,舳舻千里,旌旗蔽空,酾酒临江,横槊赋诗,固一世之雄也,而今安在哉?况吾与子,渔樵于江渚之上,侣鱼虾而友麋鹿;驾一叶之扁舟,举匏樽以相属。寄蜉蝣于天地,渺沧海之一粟。哀吾生之须臾,羡长江之无穷。挟飞仙以遨游,抱明月而长终。知不可乎骤得,托遗响于悲风。”

苏子曰:“客亦知夫水与月乎?逝者如斯,而未尝往也;盈虚者如彼,而卒莫消长也,盖将自其变者而观之,则天地曾不能以一瞬;自其不变者而观之,则物与我皆无尽也,而又何羡乎?且夫天地之间,物各有主,苟非吾之所有,虽一毫而莫取。惟江上之清风,与山间之明月,耳得之而为声,目遇之而成色,取之无禁,用之不竭,是造物者之无尽藏也,而吾与子之所共适。”

客喜而笑,洗盏更酌。肴核既尽,杯盘狼籍,相与枕藉乎舟中,不知东方之既白。

\subsubsection{节标题}
九重城阙烟尘生,千乘万骑西南行。翠华摇摇行复止,西出都门百余里。
六军不发无奈何,宛转蛾眉马前死。花钿委地无人收,翠翘金雀玉搔头。
君王掩面救不得,回看血泪相和流。黄埃散漫风萧索,云栈萦纡登剑阁。
峨嵋山下少人行,旌旗无光日色薄。蜀江水碧蜀山青,圣主朝朝暮暮情。
行宫见月伤心色,夜雨闻铃肠断声。 

